\usepackage[a4paper,top=2.5cm,bottom=2.5cm,left=3cm,right=3cm,% margins
			headheight=1.5cm,headsep=1.5em,
			footskip=2em,
			]{geometry}

% 载入常用的数学包, 符号包
\usepackage{amsmath}
\usepackage{amsfonts}
\usepackage{amssymb}
\usepackage{mathrsfs}
\usepackage{blindtext}

%% linespace 行间距,段间距等等
\usepackage{setspace}
% \usepackage{indentfirst} % then the first line of each title should start with a indent.
% 定义标题和段落样式
% 定义1.5倍行距
\renewcommand{\baselinestretch}{1.62}
\setlength{\baselineskip}{12pt}   % set the fixed value of the lineskip
\setlength\parskip{\baselineskip} % set the space between the paragraphs, set the variable \parskip \baselineskip
% parindent
\setlength{\parindent}{0pt}

% fonts (style, color, size). 
\usepackage{ctex}		 	% If you are lazy, the CTEX suit is enough.

% Chinese font
\usepackage{xeCJK}		 	% For the Chinese through XeLaTex
\setCJKmainfont{FandolSong} 	% set the mainfont of Chinese as songti. (serif) for  
\setCJKsansfont{FandolSong}	% sans serif font for \textsf
\setCJKmonofont{FandolSong}	% monospace font for \texttt
% \punctstyle{kaiming}   	% Remove the space used by symbols like comma. Special for the mainland students like us HUSTers.
\setCJKfamilyfont{song}{FandolSong} 
\newcommand{\song}{\CJKfamily{song}} %宋体 song
\setCJKfamilyfont{kai}{FandolKai} 
\newcommand{\kai}{\CJKfamily{kai}} %楷体2312  kai
\setCJKfamilyfont{hwzs}{FandolFang}
\newcommand{\hwzs}{\CJKfamily{hwzs}} %华文中宋  hwzs
\setCJKfamilyfont{hei}{FandolHei} 
\newcommand{\hei}{\CJKfamily{hei}} %黑体  hei

% English font
\usepackage{fontspec}
\setmainfont{Times New Roman}
\setsansfont{Times New Roman}
\setmonofont{Times New Roman}
% font Color 利用definecolor自己可以定义颜色
\usepackage{xcolor}
\definecolor{MSBlue}{rgb}{.204,.353,.541}
\definecolor{MSLightBlue}{rgb}{.31,.506,.741}
% font Size 
\newcommand{\xiaochuhao}{\fontsize{36pt}{\baselineskip}\selectfont}
\newcommand{\erhao}{\fontsize{21pt}{\baselineskip}\selectfont}
\newcommand{\xiaoerhao}{\fontsize{18pt}{\baselineskip}\selectfont}
\newcommand{\sanhao}{\fontsize{15.75pt}{\baselineskip}\selectfont}
\newcommand{\sihao}{\fontsize{14pt}{18pt}\selectfont}
\newcommand{\xiaosihao}{\fontsize{12pt}{18pt}\selectfont}
\newcommand{\wuhao}{\fontsize{10.5pt}{18pt}\selectfont}

%% header and footer 
\usepackage{fancyhdr} % for header and footer
% header
\newcommand{\headstyle}{
 \fancyhead[C]{ \hwzs\wuhao 华中科技大学软件课程设计}
}
% 设置页脚样式
\newcommand{\footstyle}{\fancyfoot[C]{\wuhao\thepage}
 \fancyfoot[L]{\rule[5pt]{6.7cm}{0.4pt}}
 \fancyfoot[R]{\rule[5pt]{6.7cm}{0.4pt}}
}
\pagestyle{fancy}
\fancyhf{} % 清空原有样式
\headstyle
\footstyle
% 定义一种新的格式叫做main
\fancypagestyle{main}{%
 \fancyhf{} % 清空原有样式
 \headstyle
 \footstyle
}
\renewcommand{\headrulewidth}{0.4pt}
% \renewcommand{\footrulewidth}{0.4pt}
% \renewcommand{\headrule}{\rule{\textwidth}{0.4pt}}

% set the styles of sections at all levels
\usepackage{titlesec}
\usepackage{titletoc}
\titleformat{\section}{\centering\hei\bfseries\xiaoerhao}{\thesection}{1em}{} % 在section标题编号后面加个点
\titleformat*{\subsection}{\raggedright\hei\bfseries\sihao}
\titleformat*{\subsubsection}{\raggedright\hei\bfseries\xiaosihao}
\titleformat{\paragraph}[hang]{\raggedright\hei\bfseries\xiaosihao}{\theparagraph}{1em}{}[]

% manual
% \titleformat{command}[shape]{format}{label}{sep}{before-code}[after-code]
% \titlespacing{command}{left}{before-sep}{after-sep}
% 设置新的层级subsubsubsection
\setcounter{tocdepth}{4}
\setcounter{secnumdepth}{4}

\newcommand{\sectionbreak}{\clearpage} % 小节从新的一页开始
% 根据学校要求设置新的section, subsection, subsection,  paragraph

% set the content of section and so on
\newcommand\seccontent{
	\song
	\xiaosihao % 默认五号字体, 行间距为1.5*\baselineskip
    \setlength{\parindent}{2em} % 首段缩进两个M字符
    \setlength{\parskip}{0pt}
    }
\newcommand\tabcontent{
	\song
	\wuhao % 默认五号字体, 行间距为1.5*\baselineskip
	\setlength{\parindent}{2em} % 首段缩进两个M字符
	\setlength{\parskip}{0pt}
}

% ---------------------------------------------------------------------------- %
% for the style of theorems, definitions, proofs and remarks 定义数学里面一些常用的环境
\usepackage{amsthm}
\newtheorem{thm}{\textbf{定理}}[section]
% The section in [] can be replaced by chapter or subsection
\theoremstyle{definition} 
\theoremstyle{plain}
\theoremstyle{remark}

% ---------------------------------------------------------------------------- %
% for the caption and reference 图表及公式的编号规范
\usepackage{float} 		 		  	% table figure positioning
\usepackage{caption}
\captionsetup[figure]{labelformat=default, labelsep=quad,name={图}}
\captionsetup[table]{labelformat=default,labelsep=quad,name={表}}
% 设置图表标题的计数方式
\renewcommand{\thefigure}{\ifnum \thesection>0 \thesection-\fi \arabic{figure}} % set caption label style to 2-1
\renewcommand{\thetable}{\ifnum \thesection>0 \thesection-\fi \arabic{table}} % set caption label style to 2-1
\DeclareCaptionFont{mylabelfont}{\hei\xiaosihao}
\captionsetup[figure]{font=mylabelfont}
\captionsetup[table]{font=mylabelfont}

% 设置图表的autoref的格式
\newcommand{\reffig}[1]{图 \ref{#1}}
\newcommand{\reftab}[1]{表 \ref{#1}}
% 公式的编号格式
\renewcommand\theequation{\arabic{section}-\arabic{equation}}

\usepackage{graphicx} % To include graphixs 添加图所需的包
\usepackage{booktabs} % To create three line table including the commands toprule, bottomrule, and midrule
% 使用tabularx库并定义新的左右中格式
\usepackage{tabularx}
\usepackage{makecell}
\newcolumntype{L}{X}
\newcolumntype{C}{>{\centering \arraybackslash}X}
\newcolumntype{R}{>{\raggedright \arraybackslash}X}

% ---------------------------------------------------------------------------- %
% set the style of counters
\makeatletter
\@addtoreset{footnote}{page}
\@addtoreset{figure}{section}
\@addtoreset{table}{section}
\@addtoreset{equation}{section}
\makeatother

% ---------------------------------------------------------------------------- %
% tableofcontents, listoftables and listoffigures 目录
%\renewcommand\listfigurename{插图列表}
%\renewcommand\listtablename{表格列表}
%\titlecontents{标题名}[左间距]{标题格式}{标题标志}{无序号标题}{指引线与页码}[下间距]
%\dottedcontents{section}[2.55em]{\song \xiaosihao \bfseries}{2.5em}{1em}
\usepackage{tocloft}
\renewcommand{\contentsname}{\centerline{ \hei\bfseries\xiaoerhao 目\hspace{2em}录}}
\titlecontents{section}[3em]{\song\xiaosihao\bfseries}{\contentslabel{3em}}{\hspace*{-3em}}{\normalfont\titlerule*[8pt]{.}\contentspage}
\titlecontents{subsection}[3em]{\song\xiaosihao}{\contentslabel{3em}}{\hspace*{-3em}}{\titlerule*[8pt]{.}\contentspage}
\titlecontents{subsubsection}[4em]{\song\xiaosihao}{\contentslabel{4em}}{\hspace*{-4em}}{\titlerule*[8pt]{.}\contentspage}
\titlecontents{paragraph}[5em]{\song\xiaosihao}{\contentslabel{5em}}{\hspace*{-5em}}{\titlerule*[8pt]{.}\contentspage}

% reference and citation 参考文献
\usepackage{natbib}
\renewcommand{\refname}{\centering\hei\xiaoerhao 参考文献}
\bibsep=0pt % 用来设置每个\bibitem之间的间距
% \newcommand{\upcite}[1]{\textsuperscript{\textsuperscript{\cite{#1}}}} % show citation label in the upperscript

% 使用特殊符号
\usepackage{amssymb}
\usepackage{wasysym}

% 制作tatement中的符号
\def\HUSTcheckedbox{$\Square\!\!\!\!\checkmark$}
\def\HUSTbox{$\Square$}

\newcommand{\makestatement}[2]{\statement{#1}{#2}}

\usepackage{enumitem}
\setlist{noitemsep}

\usepackage{listings} % For the code. 代码
\usepackage[bookmarks=true,colorlinks,linkcolor=black,citecolor=black,urlcolor=purple]{hyperref}
\usepackage{appendix}
\renewcommand{\appendixname}{附录}

% titlepage
\usepackage{titling}
% 重置命令 maketitle
\renewcommand{\maketitle}{
	\def\HUSTtitlelength{12em}
 	\begin{titlepage}
		\begin{center}
			\vspace*{0em}
			\includegraphics[height=1.61cm]{fig/HUSTlogo.eps}\\
			\vspace*{4em}
			{\xiaochuhao \hwzs \bfseries 软件课程设计报告(个人)}\\
			\vspace*{6em}
			{\erhao \kai \bfseries \thetitle}

			\vspace*{6em}
			{\sanhao \hwzs
				\renewcommand\arraystretch{2}
				\begin{tabular}{lc}
					\makebox[4em][s]{院 \hfill 系} &
					\underline{\makebox[\HUSTtitlelength]{\school}} \\
     	              \makebox[4em][s]{专业} &
					\underline{\makebox[\HUSTtitlelength]{\major}} \\
     	              \makebox[4em][s]{班级} &
					\underline{\makebox[\HUSTtitlelength]{\class}} \\
					\makebox[4em][s]{姓 \hfill 名} &
					\underline{\makebox[\HUSTtitlelength]{\theauthor}} \\
					\makebox[4em][s]{学 \hfill 号} &
					\underline{\makebox[\HUSTtitlelength]{\studentId}} \\
					\makebox[4em][s]{指导教师} &
					\underline{\makebox[\HUSTtitlelength]{\instructor}} \\
 					\makebox[4em][s]{日\hfill 期} &
					\underline{\makebox[\HUSTtitlelength]{\thedate}} \\
			  \end{tabular}
		    }

			\vspace{4em}
			{\sanhao \hwzs \thedate}
		\end{center}
	\end{titlepage}
}

\newcommand{\hongzifuzhu}[1]{\textcolor{red}{\kai \wuhao(#1)}}